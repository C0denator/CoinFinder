\documentclass{article}
\usepackage[a4paper, left=3cm, right=3cm, top=2.5cm, bottom=2.5cm]{geometry} %Seitenränder
\usepackage[utf8]{inputenc} %Umlaute
\usepackage[T1]{fontenc} %Umlaute 
\usepackage{eurosym} %Eurozeichen
\usepackage[colorlinks=true, linkcolor=black, urlcolor=blue]{hyperref} %Links
\usepackage[ngerman]{babel} % für Deutsche Silbentrennung
\usepackage[autostyle=true, german=quotes]{csquotes} %Anführungszeichen
\usepackage{enumitem} %Anpassen von Listen
\usepackage{graphicx} %Bilder
\usepackage{subcaption} %Bilder nebeneinander
\usepackage{listings}
\usepackage[dvipsnames]{xcolor}

% Farben für JavaScript-Code definieren
\definecolor{lightgray}{rgb}{.9,.9,.9}
\definecolor{darkgray}{rgb}{.4,.4,.4}
\definecolor{darkGreen}{rgb}{0.0, 0.5, 0.0}

% JavaScript als neue Sprache definieren
\lstdefinelanguage{JavaScript}{
  keywords={abstract, any, as, boolean, break, case, catch, class, console, 
  const, continue, debugger, declare, default, delete, do, else, enum, export, 
  extends, false, finally, for, from, function, get, if, implements, import, in, 
  infer, instanceof, interface, keyof, let, module, namespace, never, new, null, 
  number, object, package, private, protected, public, readonly, require, return, 
  set, static, string, super, switch, symbol, this, throw, true, try, type, typeof, 
  undefined, unique, unknown, var, void, while, with, yield},
  morecomment=[l]{//},
  morecomment=[s]{/*}{*/},
  morestring=[b]',
  morestring=[b]",
  ndkeywords={class, export, boolean, throw, implements, import, this},
  sensitive=true
}

% Einstellungen für JavaScript-Code
\lstset{
   language=JavaScript,
   backgroundcolor=\color{lightgray},
   extendedchars=true,
   basicstyle=\footnotesize\ttfamily,
   showstringspaces=false,
   showspaces=false,
   numbers=left,
   numberstyle=\footnotesize,
   numbersep=9pt,
   tabsize=2,
   breaklines=true,
   showtabs=false,
   captionpos=b,
   literate=
  {á}{{\'a}}1 {é}{{\'e}}1 {í}{{\'i}}1 {ó}{{\'o}}1 {ú}{{\'u}}1
  {Á}{{\'A}}1 {É}{{\'E}}1 {Í}{{\'I}}1 {Ó}{{\'O}}1 {Ú}{{\'U}}1
  {à}{{\`a}}1 {è}{{\`e}}1 {ì}{{\`i}}1 {ò}{{\`o}}1 {ù}{{\`u}}1
  {À}{{\`A}}1 {È}{{\`E}}1 {Ì}{{\`I}}1 {Ò}{{\`O}}1 {Ù}{{\`U}}1
  {ä}{{\"a}}1 {ë}{{\"e}}1 {ï}{{\"i}}1 {ö}{{\"o}}1 {ü}{{\"u}}1
  {Ä}{{\"A}}1 {Ë}{{\"E}}1 {Ï}{{\"I}}1 {Ö}{{\"O}}1 {Ü}{{\"U}}1
  {â}{{\^a}}1 {ê}{{\^e}}1 {î}{{\^i}}1 {ô}{{\^o}}1 {û}{{\^u}}1
  {Â}{{\^A}}1 {Ê}{{\^E}}1 {Î}{{\^I}}1 {Ô}{{\^O}}1 {Û}{{\^U}}1
  {ã}{{\~a}}1 {ẽ}{{\~e}}1 {ĩ}{{\~i}}1 {õ}{{\~o}}1 {ũ}{{\~u}}1
  {Ã}{{\~A}}1 {Ẽ}{{\~E}}1 {Ĩ}{{\~I}}1 {Õ}{{\~O}}1 {Ũ}{{\~U}}1
  {œ}{{\oe}}1 {Œ}{{\OE}}1 {æ}{{\ae}}1 {Æ}{{\AE}}1 {ß}{{\ss}}1
  {ű}{{\H{u}}}1 {Ű}{{\H{U}}}1 {ő}{{\H{o}}}1 {Ő}{{\H{O}}}1
  {ç}{{\c c}}1 {Ç}{{\c C}}1 {ø}{{\o}}1 {Ø}{{\O}}1 {å}{{\r a}}1 {Å}{{\r A}}1
  {€}{{\euro}}1 {£}{{\pounds}}1 {«}{{\guillemotleft}}1
  {»}{{\guillemotright}}1 {ñ}{{\~n}}1 {Ñ}{{\~N}}1 {¿}{{?`}}1 {¡}{{!`}}1 
}

\lstdefinestyle{JavaScript}{
  language=JavaScript,
  keywordstyle=\color{blue}\bfseries,
  ndkeywordstyle=\color{darkgray}\bfseries,
  identifierstyle=\color{black},
  commentstyle=\color{darkGreen}\ttfamily,
  stringstyle=\color{Bittersweet}\ttfamily,
}

\lstdefinestyle{HTML}{
  language=HTML,
  keywordstyle=\color{blue}\bfseries,
  ndkeywordstyle=\color{darkgray}\bfseries,
  identifierstyle=\color{black},
  commentstyle=\color{darkGreen}\ttfamily,
  stringstyle=\color{Bittersweet}\ttfamily,
}

\lstdefinelanguage{CSS}{
    sensitive=true, % CSS ist case-sensitive
    morekeywords={
        align-content, align-items, align-self, all, animation, animation-delay, 
        animation-direction, animation-duration, animation-fill-mode, 
        animation-iteration-count, animation-name, animation-play-state, 
        animation-timing-function, backface-visibility, background, 
        background-attachment, background-blend-mode, background-clip, 
        background-color, background-image, background-origin, 
        background-position, background-repeat, background-size, border, 
        border-bottom, border-bottom-color, border-bottom-left-radius, 
        border-bottom-right-radius, border-bottom-style, border-bottom-width, 
        border-collapse, border-color, border-image, border-image-outset, 
        border-image-repeat, border-image-slice, border-image-source, 
        border-image-width, border-left, border-left-color, border-left-style, 
        border-left-width, border-radius, border-right, border-right-color, 
        border-right-style, border-right-width, border-spacing, border-style, 
        border-top, border-top-color, border-top-left-radius, 
        border-top-right-radius, border-top-style, border-top-width, 
        border-width, bottom, box-shadow, box-sizing, caption-side, clear, 
        clip, color, column-count, column-fill, column-gap, column-rule, 
        column-rule-color, column-rule-style, column-rule-width, column-span, 
        column-width, columns, content, counter-increment, counter-reset, 
        cursor, direction, display, empty-cells, filter, flex, flex-basis, 
        flex-direction, flex-flow, flex-grow, flex-shrink, flex-wrap, float, 
        font, font-family, font-feature-settings, font-kerning, font-size, 
        font-size-adjust, font-stretch, font-style, font-variant, 
        font-variant-caps, font-variant-ligatures, font-variant-numeric, 
        font-weight, grid, grid-area, grid-auto-columns, grid-auto-flow, 
        grid-auto-rows, grid-column, grid-column-end, grid-column-gap, 
        grid-column-start, grid-gap, grid-row, grid-row-end, grid-row-gap, 
        grid-row-start, grid-template, grid-template-areas, 
        grid-template-columns, grid-template-rows, hanging-punctuation, 
        height, hyphens, justify-content, left, letter-spacing, line-break, 
        line-height, list-style, list-style-image, list-style-position, 
        list-style-type, margin, margin-bottom, margin-left, margin-right, 
        margin-top, max-height, max-width, min-height, min-width, 
        mix-blend-mode, object-fit, object-position, opacity, order, orphans, 
        outline, outline-color, outline-offset, outline-style, outline-width, 
        overflow, overflow-wrap, overflow-x, overflow-y, padding, 
        padding-bottom, padding-left, padding-right, padding-top, 
        page-break-after, page-break-before, page-break-inside, perspective, 
        perspective-origin, pointer-events, position, quotes, resize, 
        right, scroll-behavior, tab-size, table-layout, text-align, 
        text-align-last, text-combine-upright, text-decoration, 
        text-decoration-color, text-decoration-line, text-decoration-style, 
        text-indent, text-justify, text-orientation, text-overflow, 
        text-shadow, text-transform, top, transform, transform-origin, 
        transform-style, transition, transition-delay, 
        transition-duration, transition-property, transition-timing-function, 
        unicode-bidi, user-select, vertical-align, visibility, 
        white-space, widows, width, word-break, word-spacing, word-wrap, 
        writing-mode, z-index
    },
    morecomment=[l]{/*}, % Kommentare beginnen mit /* und enden mit */
    morestring=[b]", % Zeichenketten mit " und '
    morestring=[b]',
    keywordstyle=\color{blue}\bfseries, % Schlüsselwörter fett und blau
    identifierstyle=\color{black}, % IDs und Klassen schwarz
    commentstyle=\color{green}\itshape, % Kommentare grün und kursiv
    stringstyle=\color{red}, % Zeichenketten rot
    sensitive=true % Groß-/Kleinschreibung beachten
}
%style für css
\lstdefinestyle{CSS}{
  language=CSS,
  keywordstyle=\color{blue}\bfseries,
  ndkeywordstyle=\color{darkgray}\bfseries,
  identifierstyle=\color{black},
  commentstyle=\color{darkGreen}\ttfamily,
  stringstyle=\color{Bittersweet}\ttfamily,
}

   %Code-Listings für JavaScript (ausgelagert in Listings.tex)

\renewcommand{\familydefault}{\sfdefault} %Schriftart auf serifenlos ändern
\renewcommand{\baselinestretch}{1.15} %Zeilenabstand
\setlength{\parskip}{0.8em} %Absatzabstand
%\setlength{\parindent}{0pt} %kein Einrücken bei neuem Absatz
\title{Devlog - Computer Vision mit openCV.js}
\author{Mattis Thieme}
\date{November 2024}

\begin{document}

\maketitle \newpage

\renewcommand*\contentsname{Inhaltsverzeichnis} %Überschrift des Inhaltsverzeichnisses ändern
\tableofcontents \newpage

\section{Einleitung}
Herzlich Willkommen! 

Probleme der Bilderkennung und Bildverarbeitung gewinnen immer mehr an Relevanz. Kein Wunder: Mit der stetigen Verbreitung von mobilen Endgeräten haben immer mehr Systeme Zugriff auf eine (mittlerweile sehr hochwertige) Kamera. Die Möglichkeiten, die sich dadurch ergeben, sind vielfältig: über das Scannen von Dokumenten, hin zur Klassifizierung von Tieren und Pflanzen bis hin zum industriellen Einsatz in der Qualitätskontrolle.

Zusätzlich geht der Trend auch in die Richtung von Onlineservices und Webanwendungen. Programme die früher noch installiert werden mussten, laufen nun direkt im Browser des Clients - entweder als online gehostete Webseite oder lokal in Form einer Cross-Plattform-Desktop-Anwendung. Hauptgrund ist natürlich die Plattformunabhängigkeit und die damit verbundene Reduktion von Entwicklungskosten und Zeit. 

Aber wie sieht es mit Bildverarbeitung und Bilderkennung im Webkontext aus? Wenn Probleme im Bereich der Bildverarbeitung gelöst werden sollen, fällt die Wahl häufig auf OpenCV. Und das nicht ohne Grund: OpenCV bietet ein riesiges Spektrum an Funktionen und Algorithmen, von einfachen Bildoperationen hin zu ausgereiften Algorithmen der Gesichtserkennung, Bildsegmentierung und Objekterkennung. Auch Maschinelles Lernen und Deep Learning sind im Funktionsumfang von OpenCV enthalten. 

Jedoch wurde OpenCV nun einmal in C++ geschrieben und kann somit nicht direkt auf Clientseite implementiert werden. Aber nicht immer ist die Option sinnvoll, Aufgaben der Bildverarbeitung auf der Seite des Servers lösen zu lassen. Nicht immer steht die benötigte Rechenleistung zur Verfügung, auch die Latenz kann letztendlich je nach Anwendungsfalls ein Problem darstellen. Somit stellt sich einem die folgende Frage: Ist es möglich OpenCV auch im Browser des Clients zu lösen? Die Antwort: Ja! Mit openCV.js.

\subsection{Motivation}
Aber warum der Blog? Einst musste ich (wie du vielleicht auch) ein Problem der Bildverarbeitung lösen, welches zwingend im Webkontext stattfinden sollte. Die Anforderungen waren klar: die Bildverarbeitung sollte direkt im Browser des Clients stattfinden, ohne dass der Nutzer eine zusätzliche Software installieren muss. 

Die Bibliothek openCV war natürlich die erste Wahl, jedoch stand ich nun genau vor diesem Problem: wie bekomme ich openCV, eine Bibliothek, die ursprünglich in C++ geschrieben ist, in meiner Webanwendung zum Laufen? Meine Lösung war natürlich openCV.js.

Als relativ neuer Bestandteil des openCV-Projektes, ist openCV.js eine JavaScript-Portierung der OpenCV-Bibliothek. Sie ermöglicht es, OpenCV-Funktionen direkt im Browser auszuführen, ohne dass der Nutzer eine zusätzliche Software installieren muss. Somit können wir die volle Bandbreite der OpenCV-Funktionen nutzen, ohne auf die Vorteile einer Webanwendung verzichten zu müssen.

Obwohl es sich zunächst als die beste Lösung angehört hat, war die Nutzung von openCV.js jedoch kein Selbstläufer. Wie du wahrscheinlich bereits mit Schrecken festgestellt hast, ist die offizielle Dokumentation nur für die C++ Schnittstelle geschrieben, und Tutorials für openCV.js sind leider rar gesät. Die JavaScript-SChnittstelle von openCV ist nun mal die neuste Änderung des mittlerweile 20 Jahre alten Projektes und somit noch nicht so ausgereift und erprobt wie die anderen Schnittstellen.

Genau aus diesen Gründen habe ich mich für das Schreiben dieses Devblogs entschieden. Ich möchte meine Erfahrungen und Erkenntnisse teilen, um anderen Entwicklern zu helfen, die vor dem gleichen Problem stehen. Ich möchte zeigen, dass es gar nicht so kompliziert ist, openCV.js in einer Webanwendung zu nutzen und wie mächtig die Bibliothek tatsächlich ist. Ja es gibt ein paar Eigenheiten und Macken, aber genau diese werde ich erläutern, damit du nicht die gleichen Fehler machst wie sie ich einst gemacht habe.

\subsection{Zielsetzung}
Kommen wir nun zu den eigentlichen Fragen: Was möchte ich erreichen? Was möchte ich entwickeln? Wie bereits oben erwähnt, möchte ich eine Webanwendung erstellen, welche durch Nutzung von openCV.js Probleme der Bildverarbeitung und Bilderkennung lösen kann. Das konkrete Problem sollte dabei sowohl häufig genutzte Methoden der Bildverarbeitung abdecken, als auch leicht verständlich und reproduzierbar sein. Ich möchte, dass du die Anwendung einfach nachbauen und selber ausprobieren kannst, um die Funktionsweise von openCV.js direkt durch eigene Erfahrungen zu erlernen.

Nach einigen Überlegungen bin ich auf die Idee gekommen, eine Webanwendung zu entwickeln, welche Münzen erkennen und deren Wert bestimmen kann. Warum ausgerechnet Münzen? Zum einen ist es ein angemessen komplexes Problem, welches gleich mehrere unterschiedliche Aspekte der Bildverarbeitung abdeckt. Zum anderen ist es eine Problemstellung, welche sich leicht auf einem anderen System reproduzieren lässt. Alles was du brauchst, sind lediglich eine Webkamera und ein paar Münzen.

Interessiert? Dann lass uns anfangen! Im meinem nächsten Beitrag werde ich direkt die grundlegende Webseitenstruktur und die Einbindung von openCV.js erläutern. 

Bis dahin! \newpage

\section{Erstellen der Webseite}
Bevor wir mit der Implementierung irgendwelcher Bildverarbeitungsalgorithmen beginnen können, müssen wir zunächst eine Webseite mit den notwendigen Funktionalitäten erstellen.

\subsection{openCV.js}
Als aller erstes braucht es natürlich eine aktuelle Version von openCV.js. Diese kann direkt von der offiziellen openCV-Webseite heruntergeladen werden:

\href{https://docs.opencv.org/4.10.0/opencv.js}{https://docs.opencv.org/4.10.0/opencv.js}

Es handelt sich hierbei um eine mittels Emscripten kompilierte Version von OpenCV, welche in JavaScript in Form einer WebAssembly ausgeführt werden kann. Das Einbinden erfolgt wie mit einer gewöhnlichen JavaScript-Datei. Da der Programmcode in Bytecode zur Verfügung steht, erfolgt die Ausführung deutlich schneller da die Kompilierung zur Laufzeit entfällt. Dies führt jedoch auch zu dem Problem, dass IDE-Funktionalitäten wie Autovervollständigung oder Syntax-Highlighting leider nicht verfügbar sind.

\subsection{Grundaufbau der Webseite}
Nun stellt sich die Frage: Welche Funktionen soll unsere Webseite haben? Zum einen brauchen wir natürlich Zugriff auf die Kamera des Nutzers, um den Videostream zu erhalten. Zum anderen benötigen wir eine Möglichkeit, das Ergebnis der Bildverarbeitung anzuzeigen.

Mithilfe eines Video-Elementes können wir den Kamerastream von Webkameras anzeigen und auf diesen zugreifen. Für die Anzeige der Bildverarbeitungsergebnisse habe ich mich für ein Canvas-Element entschieden. Warum? - In OpenCV gibt es viele Möglichkeiten, Zeichenvorgänge auf Matrizen anzuwenden. Diese Matrizen können dann auf ein Canvas-Element gezeichnet werden (mehr dazu weiter unten). Mein Plan ist somit alle relevanten Ergebnisse direkt auf den zu verarbeitenden Frame zu zeichnen und diesen dann auf dem Canvas-Element anzuzeigen.

Nach einigen Tests habe ich mich für folgende Struktur entschieden:
\begin{lstlisting}[style=HTML]
<!DOCTYPE html>
<html lang="de">
<head>
    <meta charset="UTF-8">
    <meta name="viewport" content="width=device-width, initial-scale=1.0">
    <title>Video Player mit Canvas Overlay</title>
    <link rel="stylesheet" href="style.css">
    <script src="lib/opencv_4.10.0.js"></script>
</head>
<body>
<div id="mainContainer">
    <div id="textContainer">
        <h1>CoinFinder</h1>
        <h3>Aktueller Wert: <span id="value">0</span></h3>
    </div>
    <div class="container" id="videoContainer">
        <video id="video" width="720" height="540" autoplay muted loop></video>
        <canvas id="outputCanvas" width="720" height="540"></canvas>
    </div>
</div>
</body>
</html>
\end{lstlisting}

Damit haben wir ein Video-Element, welches den Kamerastream anzeigt und ein Canvas-Element, um die Bildverarbeitungsergebnisse anzuzeigen. In der nachfolgenden CSS-Datei sorge ich dafür, dass beide Elemente stets übereinander liegen und die gleiche Größe haben. So sieht der Nutzer immer das aktuelle Kamerabild und die Bildverarbeitungsergebnisse direkt darüber. Zusätzlich bekommt der Canvas die Eigenschaft "image-rendering: pixelated;", damit auch kleine Matrizen originalgetreu dargestellt werden.

Die CSS-Datei sieht wie folgt aus:

\begin{lstlisting}[style=CSS]
    *{
        box-sizing: border-box;
    }
    
    html, body {
        background-color: #18204d;
        color: #ffffff;
        font-family: Arial, Helvetica, sans-serif;
    }
    
    h3{
        margin: 0 0 0 1em;
    }
    
    #mainContainer{
        position: relative;
        width: 97vw;
        height: 97dvh;
        display: flex;
        flex-direction: column;
    }
    
    #textContainer{
        display: flex;
        align-items: center;
    }
    
    #textContainer > *{
        margin: 0 1em 0 1em;
    }
    
    button{
        background-color: #ffa300;
        color: #18204d;
        border: none;
        padding: 0.5em 1em;
        font-size: 1em;
        cursor: pointer;
    }
    
    #videoContainer{
        position: relative;
        height: 100%;
        width: 100%;
        flex-grow: 1;
        align-items: center;
        justify-content: center;
        display: flex;
    
    }
    
    #video, #outputCanvas {
        height: 100%;
        width: 100%;
        left: 0;
        top: 0;
        position: absolute;
        aspect-ratio: inherit;
        object-fit: contain;
    }
    
    #outputCanvas{
        image-rendering: pixelated;
    }
    
    #mainContainer{
        border: 0.5em solid #ffa300;
    }
    
    #videoContainer{
        border: 0.5em solid blue;
    }
    
\end{lstlisting}

Als nächstes braucht es nun grundlegende Methoden, um auf die Kamera des Nutzers zuzugreifen und den Videostream auf dem Video-Element anzuzeigen.

\subsection{Grundlegende openCV Methoden}
Um auf die Kamera des Nutzers zuzugreifen, kann man unter anderem ein openCV-VideoCapture-Objekt verwenden. Dieses Objekt kann entweder direkt auf ein Video-Element zugreifen oder auf eine Datei. Da wir in unserem Fall den Kamerastream verwenden wollen, greifen wir direkt auf das Video-Element zu. Für den Zugriff auf die Webkamera benötigen wir die getUserMedia()-Methode des MediaDevices-Interfaces. Um sicherzustellen, dass openCV vollständig geladen ist, warten wir zusätzlich auf das "load"-Event des Fensters und initialisieren erst dann die Kamera:

\begin{lstlisting}[style=JavaScript]
    window.addEventListener("load", function () {
        video = document.getElementById('video');
        outputCanvas = document.getElementById('outputCanvas');
        videoContainer = document.getElementById('videoContainer');

        // get the camera
        navigator.mediaDevices.getUserMedia({
            video: true,
            audio: false
        }).then(stream => {
            video.srcObject = stream;
            video.onloadedmetadata = () => {
                video.play()

                //print camera stats
                console.log("Camera resolution: " + video.videoWidth + "x" + video.videoHeight);
                console.log("Camera frame rate: " + stream.getVideoTracks()[0].getSettings().frameRate+ " fps");

                //initialize the inputMat-Matrix
                inputMat = new cv.Mat(video.height, video.width, cv.CV_8UC4);
                guiMat = new cv.Mat(video.height, video.width, cv.CV_8UC4);
                videoCapture = new cv.VideoCapture(video);

                //check if everything is loaded
                cameraLoaded = true;
                console.log("Camera loaded");
                CheckIfLoadingFinished();
            };
        }).catch(error => {
            console.error('Error accessing the camera: ', error);
        });
    };
}
\end{lstlisting}

Nachdem die Kamera initialisiert ist, kann die read()-Methode des VideoCapture-Objekts verwendet werden, um den aktuellen Frame des Videostreams in eine openCV-Matrix zu konvertieren:

\begin{lstlisting}[style=JavaScript]
let inputMat;
let guiMat;

videoCapture.read(inputMat);
\end{lstlisting}

Für die nachfolgenden Funktionen werde ich immer wieder die Variablen \texttt{inputMat} und \texttt{guiMat} verwenden. \texttt{inputMat} enthält den aktuellen Frame des Videostreams, während \texttt{guiMat} eine Kopie des Frames ist, auf welchem die Bildverarbeitungsergebnisse gezeichnet werden. 

Nun brauchen wir jedoch noch eine Methode, um eine openCV-Matrix (in meinem Fall \texttt{guiMat}) auf einem Canvas ausgeben zu können. Dafür können wir einfach die cv.imshow()-Methode verwenden, welche eine Matrix auf ein Canvas-Element zeichnet. In unserem Fall verwenden wir das outputCanvas-Element, welches über dem Video-Element liegt. Die Funktion zum Anzeigen einer openCV-Matrix sieht wie folgt aus:

\begin{lstlisting}[style=JavaScript]
function ShowMatrix(src, canvas){
    cv.imshow(canvas, src);
}
\end{lstlisting}

Hinweis zur Speicherfreigabe: openCV.js verwaltet den Speicher nicht automatisch, wie es bei JavaScript üblich ist. Das bedeutet, dass wir selbst dafür verantwortlich sind, den Speicher freizugeben, sobald wir ihn nicht mehr benötigen. Dies betrifft hauptsächlich Objekte vom Typ cv.Mat und cv.MatVector, welche wir mit der delete()-Methode freigeben können. Tun wir dies nicht, verbraucht die Anwendung mit jedem neuen Aufruf von new cv.Mat() oder cv.imread() mehr Speicher, bis irgendwann der Browsertab abstürzt. Sollte dein Programm nach einigen Sekunden oder Minuten aufhören zu funktionieren, könnte dies ein Hinweis auf ein Speicherleck sein. Schaue in diesem Fall in die Konsole nach einer entsprechenden Fehlermeldung.

Zu guter Letzt benötigen wir noch eine Hauptschleife, bei der die Bilddaten aus dem Video-Element extrahiert, die zukünftigen Bildverarbeitungsalgorithmen ausgeführt und das Ergebnis auf dem Canvas-Element anzeigt werden. Dafür können wir entweder die \texttt{requestAnimationFrame()}-Methode verwenden, welche die Ausführung der Methode automatisch der Bildschirmwiederholrate anpasst, oder wir benutzen \texttt{setTimeout()} um selber die Zeit zwischen der wiederholten Ausführung festzulegen. Ich habe zusätzlich einen Bool "loopActive" hinzugefügt, um den Loop bei Bedarf auch nur einmalig ausführen zu können. Die Funktion sieht schließlich so aus:

\begin{lstlisting}[style=JavaScript]
let waitingForAnimationFrame = false;
function mainLoop() {
    if(!loadingFinished){
        console.warn("Can't start the loop because something is not loaded yet");
        return;
    }

    waitingForAnimationFrame = false;

    console.log("--- loop started");

    videoCapture.read(inputMat);
    videoCapture.read(guiMat);

    //do something with the matrix
    
    ShowMatrix(guiMat, outputCanvas);

    if(loopActive){
        waitingForAnimationFrame = true;
        requestAnimationFrame(mainLoop);
    }

    console.log("--- loop ended");
}
\end{lstlisting}

Nun haben wir das Grundgerüst für unsere Webseite erstellt. Als nächstes können wir uns an die Implementierung der ersten Bildverarbeitungsalgorithmen wie der Kreiserkennung machen. Mehr dazu gibt es im nächsten Eintrag. Bis bald!
 \newpage

\section{Eingriffe zur Laufzeit}

\subsection{Die Bedeutung von einfachem Testing}
Meine nächste Aufgabe, die Implementierung der Kreiserkennung, erfordert eine Menge an Feintuning und Anpassungen. Jeder Parameter muss sorgfältig gewählt werden, um ein optimales Ergebnis zu erzielen. Zudem kann ich vorhinein nicht ermittelt werden, welche Parameter die besten sind. Es ist also ein iterativer Prozess, bei dem ich die Parameter anpasse, das Ergebnis betrachte und dann erneut die Parameter anpasse.

Nun stelle man sich vor, man müsste für jede kleine Änderung die Webseite neu laden, die Kamera neu ausrichten und die Münzen neu platzieren. Um diesen Prozess deutlich zu beschleunigen, habe ich mich dazu entschieden, einige Eingriffe zur Laufzeit zu implementieren. Diese Eingriffe sollen es mir ermöglichen, die Parameter der Kreiserkennung und weiterer openCV Funktionen direkt zur Laufzeit zu verändern, ohne die Webseite neu laden zu müssen, und somit schnell die besten Parameter zu finden.

\subsection{Variablen-Slider}
Meine Idee war es, für jede Variable im JavaScript-Code einen Slider zu erstellen zu können, mit dem ich schnell und einfach den Wert der Variable verändern kann. Dieser Slider soll dann den Wert der Variable anzeigen und bei Veränderung den Wert der Variable ändern.

Für die Slider habe ich die Bibliothek \href{https://refreshless.com/nouislider/}{noUISlider} verwendet.  Über die data-Attribute von den HTML-Elementen kann unkompliziert der Wertebereich und die Schrittweite des Sliders festgelegt werden.

Nun muss aber standardmäßg ein Event-Listener für jeden Slider erstellt werden, der bei Veränderung den Wert einer Variable ändert. Da ich jedoch nicht für jede Variable einen eigenen Event-Listener erstellen möchte, habe ich mich für einen anderen Ansatz entschieden. Stattdessen soll man direkt im HTML-Element des Sliders angeben können, welche Variable durch diesen Slider verändert werden soll.

Hierfür benötigen wir zum Ersten ein eigenes data-Attribute von HTML, in welchem der Name der zu verändernen Varaible angegeben wird. Zum Zweiten benötigen wir eine Funktion, die alle Slider-Elemente durchgeht und für jedes Element den Event-Listener erstellt. Anschließend braucht es noch eine Möglichkeit den String-Namen der Variable in eine Referenz auf die Variable umzuwandeln.

Die Event-Listener Funktion sieht dann wie folgt aus:

\begin{lstlisting}[style=JavaScript]
function InitSliders(){
    // Alle Slider-Container selektieren
    const sliderContainers = document.querySelectorAll('.sliderContainer');

    sliderContainers.forEach(container => {
        const sliderElement = container.querySelector('.slider');
        const valueElement = container.querySelector('.sliderValue');

        const min = parseFloat(container.dataset.min);
        const max = parseFloat(container.dataset.max);
        const step = parseFloat(container.dataset.step);
        const var1 = container.dataset.var1;
        const var2 = container.dataset.var2;
        const isRange = container.dataset.range === "true"; // Überprüft, ob Bereichsmodus aktiv ist
        //console.log("Data for slider: min: " + min + " max: " + max + " step: " + step + " var1: " + var1 + " var2: " + var2 + " isRange: " + isRange);

        // Startwerte auslesen
        let startValues = [];
        if (isRange) {
            // Bereichsmodus: Startwerte aus den <span>-Elementen lesen
            const spanValues = valueElement.querySelectorAll('span');
            startValues = Array.from(spanValues).map(span => parseFloat(span.textContent));
            if (startValues.length !== 2) {
                // Fallback: Standardwerte in der Mitte des Bereichs
                startValues = [min + (max - min) / 3, max - (max - min) / 3];
            }
        } else {
            // Einzelregler: Einzelwert auslesen
            startValues = [parseFloat(valueElement.textContent) || (min + max) / 2];
        }

        // Slider erstellen
        noUiSlider.create(sliderElement, {
            start: startValues,
            range: {
                'min': min,
                'max': max
            },
            step: step,
            connect: isRange ? true : [true, false] // Verbindet die Regler bei Range-Modus
        });

        SetSliderValueFromCookie(container.id);

        // Update-Event
        sliderElement.noUiSlider.on('update', (values, handle) => {
            if (isRange) {
                // Bereichsmodus: Werte in <span>-Elementen aktualisieren
                if(step >= 1){
                    valueElement.textContent = values.map(value => Math.round(value)).join(' - ');
                }else{
                    valueElement.textContent = values.map(value => value).join(' - ');
                }

                //TODO: Beide Variablen aktualisieren
                if(var1 !== undefined && var2 !== undefined){
                    UpdateVariable(var1, values, 0);
                    UpdateVariable(var2, values, 1);
                }else{
                    console.log("var1 or var2 is undefined. Slider will not change any variables");
                }


            } else {
                // Einzelregler: Textinhalt aktualisieren
                if(step >= 1){
                    valueElement.textContent = Math.round(values[handle]);
                }else{
                    valueElement.textContent = values[handle];
                }

                if(var1 !== undefined){
                    // Variable aktualisieren
                    UpdateVariable(var1, values, handle);
                }else{
                    console.log("var1 is undefined. Slider will not change any variables");
                }

            }

            // Cookie setzen
            SetSliderCookie(container.id, values);
        });
    });

    console.log("Sliders initialized");
}
\end{lstlisting}

Die Funktion UpdateVariable() wandelt schließlich den String-Namen der Variable in eine Referenz um und ändert den Wert der Variable. Hierfür wird die eval()-Funktion verwendet, die den String als JavaScript-Code interpretiert.

\begin{lstlisting}[style=JavaScript]
function UpdateVariable(varName, values, handle){
    //check if variable exists
    if (eval('typeof ' + varName) === 'undefined') {
        console.warn('Variable '+varName+' does not exist');
        return;
    }

    //update variable
    if(eval('typeof ' + varName) === 'number'){
        eval(varName + ' = ' + parseFloat(values[handle]));
    }else if(eval('typeof ' + varName) === 'string'){
        eval(varName + ' = ' + values[handle]);
    }
}
\end{lstlisting}

Wie du vielleicht gesehen hast, gibt es auch noch die Funktionen SetSliderValueFromCookie() und SetSliderCookie(). Diese Funktionen dienen dazu, die Werte der Slider in Cookies zu speichern und beim Laden der Webseite wiederherzustellen. So behalten die Slider ihre eingestellten Werte auch nach einem Neuladen der Webseite.

\subsection{Cookies}

Cookies sind kleine Textdateien, die auf dem Computer des Benutzers gespeichert werden. Sie dienen dazu, Informationen über den Benutzer zu speichern, um die Benutzererfahrung zu verbessern. In meinem Fall möchte ich die Werte der Slider speichern, um sie beim Neuladen der Webseite wiederherzustellen.

Jedoch ist das Speichern von Werten in Cookies nicht ganz trivial: Pro Cookie kann nur ein einziger String mit einer maximalen Länge von 4096 Bytes gespeichert werden. Da die Werte der Slider jedoch Arrays von Zahlen sind, müssen diese erst in einen String umgewandelt werden. Hierfür habe ich mich für die JSON.stringify()-Funktion entschieden, die ein JavaScript-Objekt in einen JSON-String umwandelt. \newpage

\section{Kreiserkennung}

\subsection{Funktionsweise der HoughCircles-Methode}
Das Implementieren der Kreiserkennung in openCV.js war tatsächlich ziemlich einfach. Alles was es grundsätzlich dafür braucht ist diese Methode:

\begin{lstlisting}[style=JavaScript]
cv.HoughCircles(image, circles, method, dp, minDist, param1, param2, minRadius, maxRadius);
\end{lstlisting}

Gehen wir erstmal die einzelnen Parameter durch:
\begin{itemize}[label={}]
    \item \textbf{image} - Dies ist das Bild, in dem die Kreise gefunden werden sollen. Es muss eine openCV-Matrix sein und sollte als Graustufenbild vorliegen.
    \item \textbf{circles} - In diese Matrix werden die gefundenen Kreise gespeichert. Ja - die Methode schreibt die Kreise direkt in eine openCV-Matrix. In OpenCV können Matrizen nicht nur für die Speicherung von Bildern verwendet werden, sondern auch für die Speicherung von Vektoren. In diesem Fall wird in der Matrix \enquote{circles} in jeder Zeile ein Kreis gespeichert. Ein Kreis ist dabei ein Vektor der Form \([x, y, r]\), wobei \((x, y)\) die Koordinaten des Mittelpunkts des Kreises und \(r\) der Radius des Kreises sind.
    
    Weiter unten zeige ich eine Möglichkeit, wie die einzelnen Kreise in dieser Matrix in Objekte umgewandelt werden können. Dies macht die weitere Verarbeitung der Kreise deutlich einfacher.
    \item \textbf{method} - Dies ist eine Konstante, die angibt, welche Methode zur Kreiserkennung verwendet werden soll. In openCV.js gibt es aktuell nur eine Methode, die verwendet werden kann, nämlich \texttt{cv.HOUGH\_GRADIENT}.
    \item \textbf{dp} - Hier wird es leider etwas komplizierter. Der Parameter \texttt{dp} ist ein inverser Skalierungsfaktor, der die Genauigkeit der Kreiserkennung beeinflusst. Ein Wert von 1 bedeutet, dass die Auflösung des Eingabebildes verwendet wird. Ein Wert von 2 bedeutet, dass das Eingabebild halb so groß wie das Originalbild ist. Je kleiner der Wert, desto genauer ist die Kreiserkennung, aber auch desto langsamer. 
    \item \textbf{minDist} - Dieser Parameter gibt den minimalen Abstand zwischen den Mittelpunkten der gefundenen Kreise an. Dies ist sehr nützlich um die Anzahl an falschen Positiven zu reduzieren. 
    \item \textbf{param1} - Dieser Parameter ist spezifisch für die Methode \texttt{cv.HOUGH\_GRADIENT}. Es handelt sich hier um den oberen Schwellwert für die Kantenerkennung, welche intern durchgeführt wird. Der untere Schwellwert ist stets die Hälfte dieses Wertes.
    \item \textbf{param2} - Dieser Parameter ist ebenfalls spezifisch für die Methode \texttt{cv.HOUGH\_GRADIENT}. Er ist der Schwellwert für die Kreiserkennung. Je kleiner der Wert, desto empfindlicher ist die Kreiserkennung, d.h. ein Kreis muss weniger stark ausgeprägt sein, um trotzdem als solcher erkannt zu werden. Dies bedeutet jedoch auch, dass das mit kleinerem Wert das Risiko für falsche Kreise steigt.
\end{itemize}

Der Einfachheit halber habe ich eine eigene Methode nur für das Finden von Kreisen geschrieben, die die Methode \texttt{cv.HoughCircles} verwendet. Diese Methode gibt dann ein Array von Kreis-Objekten zurück, die ich dann weiterverarbeiten kann. Die Methode sieht so aus:

\begin{lstlisting}[style=JavaScript]
function FindCircles(inputMat, guiMat){

    //Eingabe-Matrix in Graustufen umwandeln
    cv.cvtColor(inputMat, grayMat, cv.COLOR_RGBA2GRAY);

    //reset circlesMat
    circlesMat.delete();
    circlesMat = new cv.Mat();

    //Hough-Transformation
    cv.HoughCircles(grayMat, circlesMat, cv.HOUGH_GRADIENT, dp, minRadius, param1, param2, minRadius, maxRadius);

    //create circle objects
    let foundCircles = [];
    for (let i = 0; i < circlesMat.cols; ++i) {
        let x = circlesMat.data32F[i * 3];
        let y = circlesMat.data32F[i * 3 + 1];
        let radius = circlesMat.data32F[i * 3 + 2];

        foundCircles.push(new Circle(x, y, radius));
    }

    //draw circles
    for(let i = 0; i < foundCircles.length; i++){
        DrawCircle(foundCircles[i], guiMat, [255,255,255,255]);
    }

    return foundCircles;
}
\end{lstlisting}

Auch für das Anzeigen der Kreise habe ich eine ausgelagerte Methode geschrieben. Dies wäre eigentlich nicht nötig gewesen, da für das Zeichnen von Kreisen nur eine einziger Methodenaufruf benötigt wird. Ich habe die Methode dennoch geschrieben, um die Lesbarkeit des Codes zu erhöhen. Die Methode sieht so aus:

\begin{lstlisting}[style=JavaScript]
function DrawCircle(circle, guiMat, color){
    //return if parameter is not set
    if(circle === undefined){
        console.error("Circle is undefined");
        return;
    }

    //draw circle
    cv.circle(guiMat, new cv.Point(circle.x, circle.y), circle.radius, color, 2);
}
\end{lstlisting}

Bei den Parametern \texttt{dp}, \texttt{param1} und \texttt{param2} stellte sich mir nun die Frage, welche Werte ich hier verwenden soll. Eine pauschale Antwort darauf gibt es nicht, da die optimalen Werte stark vom Bild, vom Verwendungszweck und in meinem Beispiel auch von der Kamera abhängen.

Meine nächste Aufgabe war es nun, für meinen Anwendungsfall die optimalen Werte für diese Parameter zu finden. 

\subsection{Die korrekten Parameter finden}
\subsection{Buffering der Kreise} \newpage

\section{Histogramme}

\subsection{Überlegungen}
Jetzt wo ich die Kreiserkennung implementiert habe, ist es mir möglich Münzen im Bild zu erkennen. Der nächste Schritt wäre es nun, die Münzen zu klassifizieren und zu unterscheiden. Doch wie klassifiziert man in openCV Bilder? Meine Idee war es, die Bilder als Histogramme zu speichern und zu vergleichen. Ein Histogramm ist eine graphische Darstellung der Häufigkeitsverteilung von Werten. In meinem Fall würde ich die Farbwerte der Münzen in einem Histogramm speichern und diese dann vergleichen.

\subsection{Die Vorlagen}
Damit ich die Münzen anhand ihrer Bildinformastionen klassifizieren kann, benötige ich zuerst Vorlagen, mit denen ich die Münzen vergleichen kann. Ich habe also Bilder von den Münzen gemacht und diese als Vorlagen gespeichert. Diese Vorlagenbilder sind die Referenzbilder, anhand derer ich die Münzen im Kamerabild auf Ähnlichkeit überprüfen möchte.

Hier hat sich vorallem die spiegelnde Oberfläche der Münzen als Problem herausgestellt. Ich musste viel mit dem Lichteinfallswinkel und der Beleuchtung spielen, um von jeder Münze ein gutes Bild zu bekommen. Auch der Abnutzungsgrad hat das Erstellen der Vorlagen erschwert. So musste ich vorallem bei den 50-20-10 Cent Münzen und bei den 5-2-1 Cent Münzen darauf achten, dass die Münzen stets eine ähnliche Optik ausweisen. Andernfalls könnte es passieren, dass die Spiegelung oder die Abnutzung als Unterschied interpretiert wird und nicht mehr die Ziffern auf den Münzen.

Meine Vorlagen sehen wie folgt aus:

\begin{figure}[ht]
    \centering
    \begin{subfigure}{0.23\textwidth}
        \includegraphics[width=\linewidth]{../CoinFinder/templates_2/Euro2.png}
    \end{subfigure}
    \begin{subfigure}{0.23\textwidth}
        \includegraphics[width=\linewidth]{../CoinFinder/templates_2/Euro1.png}
    \end{subfigure}
    \begin{subfigure}{0.23\textwidth}
        \includegraphics[width=\linewidth]{../CoinFinder/templates_2/Cent50.png}
    \end{subfigure}
    \begin{subfigure}{0.23\textwidth}
        \includegraphics[width=\linewidth]{../CoinFinder/templates_2/Cent20.png}
    \end{subfigure}

    \begin{subfigure}{0.23\textwidth}
        \includegraphics[width=\linewidth]{../CoinFinder/templates_2/Cent10.png}
    \end{subfigure}
    \begin{subfigure}{0.23\textwidth}
        \includegraphics[width=\linewidth]{../CoinFinder/templates_2/Cent5.png}
    \end{subfigure}
    \begin{subfigure}{0.23\textwidth}
        \includegraphics[width=\linewidth]{../CoinFinder/templates_2/Cent2.png}
    \end{subfigure}
    \begin{subfigure}{0.23\textwidth}
        \includegraphics[width=\linewidth]{../CoinFinder/templates_2/Cent1.png}
    \end{subfigure}
\end{figure}

\subsection{Erstellung der Histogramme}

\subsection{Vergleich der Histogramme}

\subsection{Erste Ergebnisse} \newpage

\section{Template Matching}
Also gut, ich habe also die Kreiserkennung implementiert. Jetzt möchte ich die Münzen klassifizieren. Doch wie mache ich das? 

\subsection{Überlegungen}
Um zur Laufzeit Münzen klassifizieren zu können, muss ich dem Programm vorher beibringen, wie diese Münzen aussehen. Dafür könnte ich für jede Münze ein einzelnes Bild abspeichern, welches dann als Vorlage für die Klassifizierung dienen könnte.

Doch wie kann man in openCV.js Bilder miteinander vergleichen und auf Ähnlichkeit prüfen? Dafür gibt es viele Mittel. Man könnte beispielsweise Histogramme für beide Bilder erstellen, und diese dann mit vorgefertigten Vergleichsalgorithmen wie der Bhattacharyya-Distanz untersuchen. Jedoch geben Histogramme nur Auskunft über die Wahrscheinlichkeitsverteilung der Farbwerte. Wichtige Informationen wie die Ziffer der Münze hätten wahrscheinlich im Histogramm keinen großen Einfluss. Es wäre dann sehr schwierig, beispielsweise den Unterschied zwischen einer 1-Cent-Münze und einer 2-Cent-Münze zu erkennen, da sich die Farben der Münzen und somit auch die Histogramme nur minimal unterscheiden würden.

Eine andere Möglichkeit ist das Template Matching. Dabei wird ein Template, also eine Vorlage, über ein Bild geschoben und die Ähnlichkeit an jedem Punkt beziehungsweise Pixel berechnet. Das Ergebnis ist eine Heatmap, die die Übereinstimmung an jedem Punkt im Bild anzeigt.

Nun möchte ich jedoch nicht eine Heatmap, die mir die Übereinstimmung für jeden Pixel anzeigt, sondern ein generelles Maß für die Ähnlichkeit des Templates zum Bild. Wenn ich vor dem Vergleich dafür sorge, dass Template und Ausgangsbild die selbe Größe haben, sollte pro Vergleich lediglich ein Wert herauskommen, der die Ähnlichkeit des Templates zum gesamten Bild beschreibt. Diesen Wert könnte ich dann dafür nutzen, um die Vorlage mit der höchsten Übereinstimmung zu finden und somit die Münze zu klassifizieren.

\subsection{Die Vorlagen}
Damit ich die Münzen anhand ihrer Bildinformationen klassifizieren kann, benötige ich also Vorlagen, mit denen ich die Münzen vergleichen kann. Ich habe demnach Bilder von den Münzen gemacht und für jede Münze ein Bild abgespeichert. Diese Vorlagenbilder sind die Referenzbilder, anhand derer ich die Münzen im Kamerabild auf Ähnlichkeit überprüfen möchte.

Hier hat sich besonders die spiegelnde Oberfläche der Münzen als Problem herausgestellt. Ich musste viel mit dem Lichteinfallswinkel und der Beleuchtung spielen, um von jeder Münze ein gutes Bild zu bekommen. Auch der Abnutzungsgrad hat das Erstellen der Vorlagen erschwert. So musste ich insbesondere bei den 50-20-10 Cent Münzen und bei den 5-2-1 Cent Münzen darauf achten, dass die Münzen stets eine ähnliche Optik ausweisen. Andernfalls könnte es passieren, dass die Spiegelung oder die Abnutzung als Unterschied interpretiert wird und nicht mehr die Ziffern auf den Münzen.

Bei jeder Vorlage nimmt die Münze das gesamte Bild ein. Sie sehen so aus:

\begin{figure}[ht]
    \centering
    \begin{subfigure}{0.23\textwidth}
        \includegraphics[width=\linewidth]{../CoinFinder/templates_2/Euro2.png}
    \end{subfigure}
    \begin{subfigure}{0.23\textwidth}
        \includegraphics[width=\linewidth]{../CoinFinder/templates_2/Euro1.png}
    \end{subfigure}
    \begin{subfigure}{0.23\textwidth}
        \includegraphics[width=\linewidth]{../CoinFinder/templates_2/Cent50.png}
    \end{subfigure}
    \begin{subfigure}{0.23\textwidth}
        \includegraphics[width=\linewidth]{../CoinFinder/templates_2/Cent20.png}
    \end{subfigure}

    \begin{subfigure}{0.23\textwidth}
        \includegraphics[width=\linewidth]{../CoinFinder/templates_2/Cent10.png}
    \end{subfigure}
    \begin{subfigure}{0.23\textwidth}
        \includegraphics[width=\linewidth]{../CoinFinder/templates_2/Cent5.png}
    \end{subfigure}
    \begin{subfigure}{0.23\textwidth}
        \includegraphics[width=\linewidth]{../CoinFinder/templates_2/Cent2.png}
    \end{subfigure}
    \begin{subfigure}{0.23\textwidth}
        \includegraphics[width=\linewidth]{../CoinFinder/templates_2/Cent1.png}
    \end{subfigure}
\end{figure}

Bevor ich nun mit dem Vergleich der Vorlagen beginne, brauche ich zunächst eine Möglichkeit um sowohl die Vorlagen als auch die daraus resultierenden Ergebnisse zu speichern. Dafür habe ich ein global verfügbares Objekt \textit{COINS} erstellt, welches die Münzen als Schlüssel enthält und als Wert ein Objekt mit den Eigenschaften \textit{diameter} und \textit{value} hat. Weitere Eigenschaften können dann zur Laufzeit den einzelnen Münzen hinzugefügt werden.

\begin{lstlisting}[style=JavaScript]
let COINS = {
    Euro2 : {
        diameter: 25.75,
        value: 2
    },
    Euro1 : {
        diameter: 23.25,
        value: 1
    },
    Cent50 : {
        diameter: 24.25,
        value: 0.5
    },
    Cent20 : {
        diameter: 22.25,
        value: 0.2
    },
    Cent10 : {
        diameter: 19.75,
        value: 0.1
    },
    Cent5 : {
        diameter: 21.25,
        value: 0.05
    },
    Cent2 : {
        diameter: 18.75,
        value: 0.02
    },
    Cent1 : {
        diameter: 16.25,
        value: 0.01
    }
}    
\end{lstlisting}

Diese Datenstruktur kann ich nun verwenden, um die Vorlagen zu speichern und ihnen eine Münze zuzuordnen. In der folgenden Methode iteriere ich über jede Münze und lese das Vorlagenbild ein:

\begin{lstlisting}[style=JavaScript]
const templatesLoaded = new Event('templatesLoaded');
let path = "../Templates/"

function InitTemplates() {
    let coinLength = Object.keys(COINS).length;
    let loadedCoins = 0;
    Object.entries(COINS).forEach(([key, value]) => {
        //is in the template folder a picture with the same name as the key?
        let img = new Image();

        img.onload = () => {
            //save image as matrix in the coin object
            COINS[key].template = cv.imread(img);

            loadedCoins++;
            if(loadedCoins === coinLength){
                document.dispatchEvent(templatesLoaded);
            }
        }

        img.onerror = () => {
            console.log("error: " + key);
        }

        img.src = path + key + ".png";

    });
}
\end{lstlisting}

Das Event \textit{templatesLoaded} wird ausgelöst, sobald alle Vorlagen geladen wurden. Somit kann ich sicherstellen, dass Programmschritt welche die Vorlagen benötigen erst ausgeführt werden, wenn die Vorlagen auch wirklich geladen wurden.
\subsection{Funktionsweise}
Für das eigentliche Template-Matching benutzte ich in openCV.js die folgende Funktion:
\begin{lstlisting}[style=JavaScript]
cv.matchTemplate(image, templ, result, method, mask);
\end{lstlisting}

Wie bereits erwähnt, berechnet diese Funktion die Übereinstimmung eines Templates mit einem Bild. Dabei wird das Template über das Bild geschoben und an jedem Punkt die Übereinstimmung berechnet. Zurück gegeben wird eine Heatmap, die die Übereinstimmung an jedem Punkt im Bild anzeigt. Glücklicherweise benötigen die Parameter nicht so viel Trial-and-Error wie bei der Kreiserkennung. Die Parameter sind wie folgt:
\begin{itemize}
    \item \textbf{image} - Das Bild, in dem das Template gesucht werden soll. Es muss als openCV-Matrix vorliegen.
    \item \textbf{templ} - Das Template, das gesucht werden soll. Es muss auch als openCV-Matrix vorliegen.
    \item \textbf{result} - Die Heatmap, in der die Übereinstimmung an jedem Punkt im Bild gespeichert wird. Auch diese muss als openCV-Matrix vorliegen.
    \item \textbf{method} - Die Methode, die zur Berechnung der Übereinstimmung verwendet werden soll. Aktuell gibt es folgende Methoden in openCV.js:
    \subitem \textbf{cv.TM\_SQDIFF} - Summe der quadrierten Differenzen (kleinere Werte bedeuten bessere Übereinstimmung)
    \subitem \textbf{cv.TM\_SQDIFF\_NORMED} - Normalisierte Summe der quadrierten Differenzen (kleinere Werte bedeuten bessere Übereinstimmung)
    \subitem \textbf{cv.TM\_CCORR} - Kreuzkorrelation (größere Werte bedeuten bessere Übereinstimmung)
    \subitem \textbf{cv.TM\_CCORR\_NORMED} - Normalisierte Kreuzkorrelation (größere Werte bedeuten bessere Übereinstimmung)
    \subitem \textbf{cv.TM\_CCOEFF} - Kreuzkorrelationskoeffizient (größere Werte bedeuten bessere Übereinstimmung)
    \subitem \textbf{cv.TM\_CCOEFF\_NORMED} - Normalisierter Kreuzkorrelationskoeffizient (größere Werte bedeuten bessere Übereinstimmung)
    \item \textbf{mask} - (optional) Eine Maske, die angibt, welche Bereiche des Bildes berücksichtigt werden sollen. In meinem Fall benutze ich keine Maske, da ich bereits vor Aufruf der Funktion unerwünschte Bereiche des Bildes ausschneide.
\end{itemize}

Der einzige Parameter welcher etwas Ausprobieren erfordert hat, war die Methode. Als erstes scheint es, als würden die Methoden \textit{cv.TM\_SQDIFF\_NORMED} und \textit{cv.TM\_CCOEFF\_NORMED} die besten Ergebnisse erzielen. 

Meine Funktion für das Template Matching sieht nun so aus:
\begin{lstlisting}[style=JavaScript]
function MatchTemplates(src, circle){
    //create result string
    let resultsString = [];
    Object.entries(COINS).forEach(([key, value]) => {
        let resultMat = new cv.Mat();

        //resize template to the size of src
        let templateResized = new cv.Mat();
        cv.resize(COINS[key].template, templateResized, COINS[key].template.size(), 0, 0, cv.INTER_AREA);

        cv.matchTemplate(src, templateResized, resultMat, cv.TM_SQDIFF_NORMED);

        //get highest value
        let minMax = cv.minMaxLoc(resultMat);
        let min = minMax.minVal;

        resultsString.push(new Result(key, min));

        templateResized.delete();
    });

    //sort results from highest to lowest
    resultsString.sort((a, b) => a.value - b.value);

    //console.dir(resultsString);

    //save bestmatch in circle
    circle.bestMatch = COINS[resultsString[0].name];
    circle.matchValue = resultsString[0].value;

    //set matchValue to 2 decimal places
    circle.matchValue = Math.round(circle.matchValue * 100) / 100;

    src.delete();
}
\end{lstlisting}

Innerhalb des main-loops wird diese Funktion nun für jeden gefundenen Kreis aufgerufen. Der Parameter "src" ist dabei eine ausgeschnittene Region des Bildes, in der sich der Kreis befindet. Der Parameter "circle" ist das Circle-Objekt, in dem die Informationen zum Kreis gespeichert sind. Die Funktion iteriert anschließend über alle Vorlagen und berechnet die Übereinstimmung für jede Vorlage. Hierbei muss beachtet werden, dass Vorlage und das zu prüfende Bild vorher auf die selbe Größe gebracht werden müssen. Die Ergebnisse werden in einem Array gespeichert und anschließend sortiert. Das beste Ergebnis wird dann im Circle-Objekt gespeichert.

Du fragst dich bestimmt wie ich an die ausgeschnittene Matrix eines jeden Kreises komme. Dafür habe ich in der \texttt{Circle}-Klasse eine eigene Methode geschrieben, die die ausgeschnittene Region des Bildes zurückgibt. Diese Methode sieht so aus:

\begin{lstlisting}[style=JavaScript]
GetImageData(src){
    let clippedImage = src.roi(new cv.Rect(this.x - this.radius, this.y - this.radius, this.radius * 2, this.radius * 2));

    return ClipCorners(clippedImage);
}
\end{lstlisting}

Dabei benutze ich die Methode \texttt{roi()}, welche es mir ermöglicht, für jede Matrix eine Region of Interest (ROI) zu definieren. Diese ROI wird dann als neue Matrix zurückgegeben. Mithilfe der XY-Koordinaten und des Radius des Kreises kann ich so in einer Matrix eine kleinere Teilmatrix erstellen, die nur den Kreis enthält. Diese Teilmatrix wird dann an die \texttt{ClipCorners()}-Funktion übergeben, welche die Ecken der Matrix abschneidet, damit das endgültige Bild nur Pixel enthält, die auch wirklich zum Kreis beziehungsweise der Münze gehören. Die Funktion \texttt{ClipCorners()} funktioniert wie folgt:

\begin{lstlisting}[style=JavaScript]
function ClipCorners(src){
    let radius = src.cols/2;
    let center = new cv.Point(src.cols/2, src.rows/2);

    let mask = cv.Mat.zeros(src.rows, src.cols, cv.CV_8UC1);
    cv.circle(mask, center, radius, [255, 255, 255, 255], -1);

    let maskedImage = new cv.Mat();
    let channels = new cv.MatVector();
    cv.split(src, channels);

    //apply mask to each channel
    for(let i = 0; i < channels.size(); i++){
        cv.bitwise_and(channels.get(i), mask, channels.get(i));
    }

    //merge channels to one image
    cv.merge(channels, maskedImage);

    //delete mats
    mask.delete();
    channels.delete();

    return maskedImage;
}
\end{lstlisting}

Diese Funktion erstellt eine Maske, die nur die Pixel enthält, die zum Kreis gehören. Um die Maske anwenden zu können, muss die Matrix in ihre einzelnen Kanäle aufgeteilt werden. Anschließend wird die Maske auf jeden Kanal angewendet, bevor die Kanäle wieder zu einem Bild zusammengefügt werden. Das Ergebnis ist ein Bild, welches nur die Pixel enthält, die zum Kreis gehören. Dieses Bild wird dann zurückgegeben.

\subsection{Exkurs: Mehrere Matrizen in einem Canvas anzeigen}
Bei wachsender Komplexität der Webanwendung ob ich mir immer wieder gewünscht eine Möglichkeit zu haben, mehrere Matrizen in einem Canvas anzuzeigen. Dies ist standardmäßig in openCV.js nicht vorgesehen, da die Funktion \texttt{cv.imshow()} immer nur eine Matrix anzeigen kann. Also habe ich mir eine eigene Funktion geschrieben, die mehrere Matrizen in einem Canvas anzeigen kann. Diese Funktion sieht so aus:

\begin{lstlisting}[style=JavaScript]
function ShowMatrices(src, canvas) {
    if (src.length === 0) {
        console.error("No src to show");
        return;
    }

    if (src.length === 1) {
        console.warn("Only one matrix to show. Using ShowMatrix instead");
        ShowMatrix(src[0], canvas);
        return;
    }

    let numberOfMatrices = src.length;

    // Berechnung der Gittergröße
    let gridCols = Math.ceil(Math.sqrt(numberOfMatrices));
    let gridRows = Math.ceil(numberOfMatrices / gridCols);

    // Maximalbreite und -höhe der Matrizen basierend auf der Größe des Canvas
    let cellWidth = canvas.width / gridCols;
    let cellHeight = canvas.height / gridRows;

    // Canvas leeren
    let ctx = canvas.getContext('2d');
    ctx.clearRect(0, 0, canvas.width, canvas.height);

    // Jede Matrix im Gitter zeichnen
    src.forEach((mat, index) => {
        // Spalte und Zeile bestimmen
        let col = index % gridCols;
        let row = Math.floor(index / gridCols);

        // Zielbereich für diese Matrix
        let x = col * cellWidth;
        let y = row * cellHeight;
        let targetSize = new cv.Size(cellWidth, cellHeight);

        // Matrix auf passende Größe skalieren
        let resizedMat = new cv.Mat();
        cv.resize(mat, resizedMat, targetSize, 0, 0, cv.INTER_AREA);

        // Matrix in RGBA umwandeln wenn es sich um eine Graustufenmatrix handelt
        if (resizedMat.channels() === 1) {
            let rgbaMat = new cv.Mat();
            cv.cvtColor(resizedMat, rgbaMat, cv.COLOR_GRAY2RGBA);
            resizedMat.delete();
            resizedMat = rgbaMat;
        }

        // Erstelle ein temporäres Canvas für diese Matrix
        let tempCanvas = document.createElement('canvas');
        let tempCtx = tempCanvas.getContext('2d');
        tempCanvas.width = resizedMat.cols;
        tempCanvas.height = resizedMat.rows;

        // In ein ImageData konvertieren und in das temporäre Canvas zeichnen
        let imageData = new ImageData(new Uint8ClampedArray(resizedMat.data), resizedMat.cols, resizedMat.rows);
        tempCtx.putImageData(imageData, 0, 0);

        // Zeichne das temporäre Canvas auf das Haupt-Canvas
        ctx.drawImage(tempCanvas, 0, 0, resizedMat.cols, resizedMat.rows, x, y, cellWidth, cellHeight);

        // Bereinige die temporären Matrizen
        resizedMat.delete();
    });
}
\end{lstlisting}

Um mehrere Bilder in einen Canvas zu zeichnen, musste ich die Größe der Bilder anpassen, damit sie in ein Grid passen. Dafür habe ich die Funktion \texttt{cv.resize()} verwendet. Anschließend müssen die Matrizen zuerst in ein temporäres Canvas gezeichnet werden, bevor sie in das Haupt-Canvas gezeichnet werden können. Die Funktion \texttt{drawImage()} von Canvas kann nur Bilder zeichnen, keine Matrizen. Daher musste ich die Matrizen in ein ImageData-Objekt umwandeln, bevor ich sie in das Canvas zeichnen konnte.
\subsection{Erste Ergebnisse}
Es hat sich gezeigt, dass der Vergleich der Vorlagen prinzipiell funktionieren kann. Jedoch ist die Genauigkeit  noch nicht zufriedenstellend. Münzen können aktuell zwar grob in die richtige Kategorie eingeordnet werden (sprich 1-2-5 Cent Münzen und 10-20-50 Cent Münzen), jedoch ist die Unterscheidung innerhalb dieser Kategorien nicht funktional. So werden aktuell nur maximal die Hälfte der Münzen korrekt zugeordnet.

Es scheint, als wären die Informationen der Ziffern auf den Münzen nicht ausreichend, um eine genaue Klassifizierung zu ermöglichen. Die Farbe einer Münze hat im Gegensatz dazu einen relativ großen Einfluss auf das Ergebnis. So werden Münzen meist der korrekten Farbkategorie zugeordnet, jedoch nicht derm korrekten Wert. Dies ist ein Hinweis darauf, dass die Farbinformationen der Münzen stärker gewichtet werden als die Forminformationen. Da sich Münzen innerhalb einer Kategorie (z.B. 1-2-5 Cent Münzen) jedoch nur in der Größe und ihrer Ziffer optisch unterscheiden, ist es schwierig eine weitere Methode der Klassifizierung zu finden. Ein Prüfen der Größe würde in meinem Anwendungsfall nicht funktionieren, da sowohl der Abstand zur Kamera als auch die Position der Münzen im Bild variabel sind. 

Eine Möglichkeit die Genauigkeit zu erhöhen wäre es, den störenden Faktor der Farbinformationen zu eliminieren, und stattdessen nur Vorlagen zu verwenden, welche nur die notwendigen Informationen enthalten. Dies könnte z.B. durch das Erstellen von Kantenbildern der Münzen erreicht werden. Solche Bilder wie sie vom Canny-Algorithmus erstellt werden, enthalten nur die Kanten des Ausgangsbildes und keine Farbinformationen, da sie rein als Schwarz-Weiß-Bilder vorliegen. Diese Kantenbilder könnten dann an Stelle der aktuellen Farbbilder als Vorlagen verwendet werden.

Jedoch gibt es noch ein weiteres Problem: die Rotation. Bereits eine kleine Drehung des Ausgangsbildes verursacht große Unterschiede in der Übereinstimmung. Das Template Matching in openCV berechnet standardmäßig nur die Übereinstimmung für das Template in der gleichen Rotation wie das Bild. Dies ist ein Problem, da die Münzen im Kamerabild in unterschiedlichen Rotationen vorkommen können. Eine Möglichkeit dieses Problem zu lösen wäre es, das Template in verschiedenen Rotationen zu speichern und für jede Rotation die Übereinstimmung zu berechnen. Dies würde jedoch die Anzahl der Vorlagen vervielfachen und einen unverhältnismäßig hohen Aufwand bedeuten. Eine andere Möglichkeit wäre es, das Ausgangsbild mehrfach zu rotieren und für jede Rotation die Übereinstimmung zu den Vorlagen berechnen. So wäre bei ausreichend kleinen Rotationsschritten sichergestellt, dass jede Münze mindestens einmal eine ähnliche Ausrichtung wie das zutreffende Template hat. Dies würde jedoch die Rechenzeit signifikant erhöhen, da bedeutend mehr Berechnungen pro Münze durchgeführt werden müssten. Zudem braucht es dafür eine Möglichkeit, alle Zwischenergebnisse der Rotationen zu speichern, um anschließend das beste Ergebnis zu wählen.

Alles in allem ein durchwachsenes Ergebnis. Die Klassifizierung der Münzen ist noch nicht zufriedenstellend, jedoch gibt es noch viele Möglichkeiten, die Genauigkeit zu erhöhen. Ich werde mich in den nächsten Tagen mit diesen Möglichkeiten beschäftigen und einige Änderungen vornehmen. Hoffentlich kann ich so die Genauigkeit der Klassifizierung noch um einiges erhöhen.  \newpage

\section{Anpassungen}

\subsection{Kantenerkennung}
Die Kantenerkennung ist ein häufig genutztes Verfahren in der Bildverarbeitung, um Objekte beziehungsweise geometrische Formen in Bildern zu erkennen. Sie ist häufig ein wichtiger Schritt für weitere Verarbeitungsschritte. So möchte auch ich die Kantenerkennung nun nutzen, um für mein Template Matching bessere Ergebnisse zu erzielen

\subsubsection{Funktionsweise der Canny-Methode}.
Um in openCV.js Kanten zu erkennen, muss der Canny-Algorithmus genutzt werden. Die Methode sieht wie folgt aus:
\begin{lstlisting}[style=JavaScript]
cv.Canny(src, dst, threshold1, threshold2, apertureSize, L2gradient);
\end{lstlisting}

\begin{itemize}
    \item \textbf{src} - Die Eingabematrix, in der die Kanten erkannt werden sollen. Für gewöhnlich wird sie vorher in ein Graustufenbild umgewandelt.
    \item \textbf{dst} - Die Ausgabematrix, in der die Kanten gespeichert werden.
    \item \textbf{threshold1} - Der erste Schwellenwert für die Hysterese. Der Algorithmus nutzt zwei Schwellenwerte, um die Kanten zu erkennen. Der erste Schwellenwert ist der niedrigere Schwellenwert, der dazu genutzt wird, um Pixel als Kanten zu markieren, wenn sie stärker sind als der Schwellenwert.
    \item \textbf{threshold2} - Der zweite Schwellenwert für die Hysterese. Der zweite Schwellenwert ist der höhere Schwellenwert, der dazu genutzt wird, um Pixel als Kanten zu markieren, wenn sie mit einem starken Pixel verbunden sind.
    \item \textbf{apertureSize} - Die Größe des Sobel-Operators, der für die Kantenerkennung genutzt wird. Ein kleinerer Wert bedeutet eine genauere Kantenerkennung, aber auch mehr Rauschen.
    \item \textbf{L2gradient} - Ein boolscher Wert, der angibt, ob die L2-Norm für die Gradienten berechnet werden soll. Wenn der Wert \textit{true} ist, wird die L2-Norm berechnet, ansonsten die L1-Norm. Die L2-Norm ist genauer, aber auch rechenaufwändiger.
\end{itemize}

Wie man sieht, arbeitet der Canny-Algorithmus mit zwei Parameter \textit{threshold1} und \textit{threshold2}, die die Schwellenwerte für die Hysterese darstellen. Für beide Parameter gilt: ein zu niedriger Wert führt zu einer hohen Anzahl an Kanten, ein zu hoher Wert zu einer geringen Anzahl an Kanten. Um die richtigen Werte zu finden, bedarf es also wieder viel "trial and error". Meine zuvor erstellten Slider werden sich hierbei als sehr nützlich erweisen.
\subsubsection{Die korrekten Parameter finden}
Während ich versuchte die korrekten Parameter zu ermittelten stellte sich relativ schnell heraus, dass ich für die Vorlagen andere Parameter benötige als für das Kamerabild. Die Vorlagen sind sehr klar und zudem deutlich größer als die Münzen im Kamerabild. Daher benötigen sie auch andere Schwellenwerte. Um dieses Problem zu lösen, habe ich für beide Anwendungsfälle ein eigene "SettingsObjekt" erstellt, in dem die Parameter für die Kantenerkennung gespeichert werden. So kann ich für die Vorlagen und das Kamerabild unterschiedliche Parameter nutzen:

\begin{lstlisting}[style=JavaScript]
class Settings{
    constructor(threshold1, threshold2, apertureSize, blurSize){
        this.threshold1 = threshold1;
        this.threshold2 = threshold2;
        this.apertureSize = apertureSize;
        this.blurSize = blurSize;
    }
}

let SettingsTemplate = new Settings(9400, 23300, 7, 7);
let SettingsLive = new Settings(79, 56, 3, 1);
\end{lstlisting}

Meine Methode für die Kantenerkennung sieht entsprechend wie folgt aus:

\begin{lstlisting}[style=JavaScript]
function DetectEdges(src, settings){
    if(!CheckForCorrectMatType(src, [MatTypes.CV_8UC4, MatTypes.CV_8UC1])) return null;

    //create output mat
    let edgesMat = new cv.Mat();

    //create gray mat
    let grayMat = new cv.Mat();

    if(src.type() === MatTypes.CV_8UC4){
        cv.cvtColor(src, grayMat, cv.COLOR_RGBA2GRAY, 0);
    }else if(src.type() === MatTypes.CV_8UC1) {
        //clone the matrix
        console.warn("src is already a gray matrix. Cloning it");
        grayMat = src.clone();
    }else{
        console.error("Unsupported type of src-matrix: " + GetMatrixType(src));
        return null;
    }

    if(settings === undefined){
        console.warn("No settings provided. Using template settings");
        settings = SettingsTemplate;
    }

    //clamp apertureSize to 3, 5 or 7
    let allowedValues = [3, 5, 7];
    settings.apertureSize = allowedValues.includes(settings.apertureSize)
        ? settings.apertureSize
        : allowedValues.reduce((closest, current) =>
            Math.abs(current - settings.apertureSize) < Math.abs(closest - settings.apertureSize) ? current : closest
        );

    //gaussian blur
    allowedValues = [1, 3, 5, 7];
    settings.blurSize = allowedValues.includes(settings.blurSize)
        ? settings.blurSize
        : allowedValues.reduce((closest, current) =>
            Math.abs(current - settings.blurSize) < Math.abs(closest - settings.blurSize) ? current : closest
        );
    cv.GaussianBlur(grayMat, grayMat, new cv.Size(settings.blurSize, settings.blurSize), 0, 0, cv.BORDER_DEFAULT);

    //detect edges
    cv.Canny(grayMat, edgesMat, settings.threshold1, settings.threshold2, settings.apertureSize, L2gradient);

    //delete mats
    grayMat.delete();

    return edgesMat;
}
\end{lstlisting}

Der Großteil der Funktion stellt lediglich sicher, dass alle Parameter einen gültigen Wert haben. So sind für \textit{apertureSize} und \textit{blurSize} nur die Werte 3, 5 und 7 bzw. 1, 3, 5 und 7 erlaubt. Sollte ein anderer Wert übergeben werden, wird der nächstgelegene verwendet. Vor der eigentlichen Kantenerkennung wird zudem eine Gaußsche Unschärfe auf das Graustufenbild angewendet. Dies kann mitunter helfen, das Rauschen zu reduzieren und die Kantenerkennung zu verbessern. Auch dieser Wert muss jedoch wie alle anderen feinfühlig per Slider eingestellt werden.

Diese Methode wird nun sowohl für jede Vorlage als auch für das Kamerabild aufgerufen. Das Template Matching arbeitet somit nicht mehr mit den Originalbildern, sondern mit den Kantenbildern. Dies sollte die Ergebnisse deutlich verbessern.

Wie dir vielleicht aufgefallen ist, habe ich in der Methode einen weiteren Check eingebaut, der überprüft, ob die übergebene Matrix ein gültiger Typ ist. Dies ist die Folge eines Problems, welches sich mir bei der Implementierung des Template Matchings stellte. Dazu mehr im nächsten Abschnitt.

\subsection{Neues Problem: Mat-Types}
Wie mir ja bereits seit einiger Zeit bekannt ist, wird in openCV die Matrixklasse \textit{Mat} genutzt, um Bilder zu speichern und zu verarbeiten. Diese Klasse hat jedoch eine Eigenschaft, die mir bislang noch nicht aufgefallen ist: den \textit{type()}. So hat nämlich jede Matrix einen Typ, der angibt, wie die Matrix aufgebaut ist. Leider wird dieser Typ nur als Integer zurückgegeben, was zunächst nicht gerade hilfreich ist. Praktischerweise gibt es im Internet jedoch Listen, in denen die Integerwerte den Typen zugeordnet werden können. So habe ich mir eine solche Methode erstellt, die mir anhand des Integerwertes den Typ der Matrix zurückgibt:

\begin{lstlisting}[style=JavaScript]
function GetMatrixType(mat) {
    const types = {
        0: "CV_8UC1",  1: "CV_8SC1",  2: "CV_16UC1",  3: "CV_16SC1",
        4: "CV_32SC1", 5: "CV_32FC1", 6: "CV_64FC1",
        8: "CV_8UC2",  9: "CV_8SC2",  10: "CV_16UC2", 11: "CV_16SC2",
        12: "CV_32SC2",13: "CV_32FC2",14: "CV_64FC2",
        16: "CV_8UC3", 17: "CV_8SC3", 18: "CV_16UC3", 19: "CV_16SC3",
        20: "CV_32SC3",21: "CV_32FC3",22: "CV_64FC3",
        24: "CV_8UC4", 25: "CV_8SC4", 26: "CV_16UC4", 27: "CV_16SC4",
        28: "CV_32SC4",29: "CV_32FC4",30: "CV_64FC4"
    };
    
    return types[mat.type] || `Unknown type: ${mat.type}`;
}
\end{lstlisting}

\subsection{Matrizen rotieren und transformieren}
Für meine MatchTemplate-Methode habe ich die Anforderung eine openCV-Matrix um einen beliebigen Winkel zu rotieren. Leider gibt es in openCV.js keine Methode, die dies direkt ermöglicht. Daher habe ich mir eine eigene Methode erstellt, die eine Matrix um einen beliebigen Winkel rotiert:

\begin{lstlisting}[style=JavaScript]
function RotateMatrix(src, dist, angle){

    let center = new cv.Point(src.cols / 2, src.rows / 2);
    let interpolation = cv.INTER_LINEAR;
    let borderMode = cv.BORDER_CONSTANT;
    let borderValue = new cv.Scalar(0, 0, 0, 255);

    //Rotationsmatrix erstellen
    let rotationMatrix = cv.getRotationMatrix2D(center, angle, 1);

    //Matrix rotieren
    cv.warpAffine(src, dist, rotationMatrix, new cv.Size(src.cols, src.rows), interpolation, borderMode, borderValue);

    //Matrix freigeben
    rotationMatrix.delete();
}
\end{lstlisting}

Für das Rotieren benutze ich die cv-eigene Methode cv.warpAffine. Sie benötigt folgende Parameter:

\begin{itemize}
    \item \textbf{src} - Die Eingabematrix, die rotiert werden soll.
    \item \textbf{dist} - Die Ausgabematrix, in der das rotierte Bild gespeichert wird.
    \item \textbf{angle} - Die Rotationsmatrix, die sowohl Grad als auch Mittelpunkt der Rotation enthält.
    \item \textbf{dsize} - Die Größe der Ausgabematrix. In meinem Fall ist sie gleich groß wie die Eingabematrix.
    \item \textbf{interpolation} - Die Interpolationsmethode, die genutzt werden soll. In meinem Fall nutze ich die LINEAR-Interpolation.
    \item \textbf{borderMode} - Der Randmodus, der genutzt werden soll. Er entscheidet, wie die Pixel am Rand des Bildes behandelt werden. In meinem Fall nutze ich den CONSTANT-Modus, der die Pixel mit einer festen Farbe füllt.
    \item \textbf{borderValue} - Die Farbe, mit der die Pixel am Rand gefüllt werden. In meinem Fall zeichne ich alle Pixel außerhalb des Bildes schwarz.
\end{itemize}

\subsection{Asynchronität und Ladebalken}
Da ich nun jeden Kreis mehrfach drehe und die Kantenerkennung für jedes Bild neu berechnen muss, dauert der Prozess des Template Matchings sehr lange. Der ursprüngliche Plan, die Methode in einer Schleife aufzurufen und das Ergebnis in Echtzeit anzuzeigen, ist somit nicht mehr möglich. Daher habe ich mich dazu entschieden, die Methode asynchron zu machen und einen Ladebalken einzufügen, der den Fortschritt anzeigt.

Ein Ladebalken kann in OpenCV relativ leicht mittels einigen Rechtecken erstellt werden:

\begin{lstlisting}[style=JavaScript]
function DrawLoadingBar(dist, value){
    //clamp value between 0 and 1
    value = Math.min(1, Math.max(0, value));

    let outerMargin = 10;
    let outerHeight = 15;
    let innerMargin = 4;
    let startPosition = new cv.Point(outerMargin, dist.rows-outerMargin-outerHeight);
    let endPosition = new cv.Point(dist.cols-outerMargin, dist.rows-outerMargin);

    //draw outer frame
    cv.rectangle(dist, startPosition, endPosition, [255, 255, 255, 255], -1);

    //draw inner frame
    let innerStartPosition = new cv.Point(startPosition.x + innerMargin, startPosition.y + innerMargin);
    let innerEndPosition = new cv.Point(endPosition.x - innerMargin, endPosition.y - innerMargin);
    cv.rectangle(dist, innerStartPosition, innerEndPosition, [0, 0, 0, 255], -1);

    //draw progress bar
    let progressEndPosition = new cv.Point(innerStartPosition.x + (innerEndPosition.x - innerStartPosition.x) * value, innerEndPosition.y);
    cv.rectangle(dist, innerStartPosition, progressEndPosition, [0, 175, 0, 255], -1);
}
\end{lstlisting}

Die Methode benötigt die Ausgabematrix, in der der Ladebalken gezeichnet werden soll, und den Wert, der den Fortschritt angibt. Der Wert muss zwischen 0 und 1 liegen. Der Ladebalken besteht aus einem weißen Rahmen, einem schwarzen Hintergrund und einem grünen Balken, der den Fortschritt anzeigt.

Jedoch darf man nicht vergessen, dass JavaScript standardmäßig single-threaded ist. Das bedeutet, dass der Browser während der Berechnung des Template Matchings nicht mehr reagiert. Um dies zu verhindern, muss dem Interpreter die Möglichkeit gegeben werden, zwischendurch von der Aufgabe abzuspringen und andere Aufgaben, wie das Zeichnen des Ladebalkens und das Aktualisieren der UI zu erledigen. Dies kann mittels der \textit{setTimeout}-Methode erreicht werden. Wenn man jedoch nicht die Methode wieder von vorne ausführen möchte, sondern direkt in der nächste Programmzeile fortfahren möchte, muss man die Methode mit einem Promise verbinden:

\begin{lstlisting}[style=JavaScript]
await new Promise(r => setTimeout(r, 0));
\end{lstlisting}

Diese Zeile bewirkt, dass der Interpreter die Methode verlässt und erst nach Ablauf der Zeit, die in \textit{setTimeout} übergeben wird, wieder zurückkehrt. Da die Zeit auf 0 gesetzt ist, wird die Methode direkt im nächsten Frame fortgesetzt. Dies gibt dem Browser jedoch die Möglichkeit, andere Aufgaben zu erledigen. \newpage

\section{Ergebnisse}

\subsection{Erste Ergebnisse}

\subsection{Zuschneiden der Bilder}

\subsection{Weitere Anpassungen}

\subsection{Finale Ergebnisse} \newpage

\section{Fazit}

\subsection{Zusammenfassung}
Die Möglichkeit von openCV.js, OpenCV-Funktionen direkt im Browser des Clients auszuführen, eröffnet eine Vielzahl von neuen Anwendungsmöglichkeiten. Rechenintensive Bildverarbeitungsalgorithmen müssen nun nicht mehr auf einem Server laufen, wodurch Kapazitäten frei werden und für andere Aufgaben genutzt werden können. Immer mehr Onlinebesuche finden auf mobilen Geräten statt - die Verwendung der eingebauten Kamera für Bildverarbeitungsaufgaben könnte somit in Zukunft eine wichtige Rolle spielen. 

\subsection{Die größten Probleme}
Es hat sich gezeigt, dass mit openCV sehr anspruchsvolle Probleme in der Bildverarbeitung gelöst werden können. Je nach Umfang und Komplexität des Problems kann die Implementierung jedoch sehr aufwendig sein. Die größten Probleme, die ich während der Implementierung hatte, waren die folgenden:

\begin{itemize}
    \item \textbf{Dokumentation} - Speziell für OpenCV.js ist die Dokumentation sehr spärlich. Die offizielle Dokumentation ist nur für C++ verfügbar und die JavaScript-Tutorials sind oft nicht ausreichend. Zudem erschwert die Emscripten-Übersetzung den Einstieg, da in der IDE keine Autovervollständigung für die OpenCV-Funktionen verfügbar ist. Es gibt keine leichte Möglichkeit, alle Funktionen und Parameter in Erfahrung zu bringen, was die Implementierung erschwert.
    \item \textbf{Umgebungsbedingungen} - Viele Methoden der Bilderkennung haben sich als sehr empfindlich gegenüber den Umgebungsbedingungen herausgestellt. So haben beispielsweise die Lichtverhältnisse und die Kameraqualität einen großen Einfluss auf die Ergebnisse. Vor allem das Licht stellt ein nur schwer zu lösendes Problem dar, welches nur schwer "genormt" werden kann. 
    \item \textbf{Parameter} - Die Parameter der OpenCV-Funktionen sind oft sehr spezifisch und müssen genau auf den Anwendungsfall abgestimmt werden. So kann es sein, dass eine Funktion bei falscher Einstellung gar nicht oder nur sehr schlechte Ergebnisse liefert. Das Finden der richtigen Parameter hat sich als sehr zeitaufwendig herausgestellt.
\end{itemize}

\subsection{Ausblick} \newpage

\end{document}