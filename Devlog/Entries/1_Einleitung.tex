\section{Einleitung}
Herzlich Willkommen! 

Probleme der Bilderkennung und Bildverarbeitung gewinnen immer mehr an Relevanz. Kein Wunder: Mit der stetigen Verbreitung von mobilen Endgeräten haben immer mehr Systeme Zugriff auf eine (mittlerweile sehr hochwertige) Kamera. Die Möglichkeiten, die sich dadurch ergeben, sind vielfältig: über das Scannen von Dokumenten, hin zur Klassifizierung von Tieren und Pflanzen bis hin zum industriellen Einsatz in der Qualitätskontrolle.

Zusätzlich geht der Trend auch in die Richtung von Onlineservices und Webanwendungen. Programme die früher noch installiert werden mussten, laufen nun direkt im Browser des Clients - entweder als online gehostete Webseite oder lokal in Form einer Cross-Plattform-Desktop-Anwendung. Hauptgrund ist natürlich die Plattformunabhängigkeit und die damit verbundene Reduktion von Entwicklungskosten und Zeit. 

Aber wie sieht es mit Bildverarbeitung und Bilderkennung im Webkontext aus? Wenn Probleme im Bereich der Bildverarbeitung gelöst werden sollen, fällt die Wahl häufig auf OpenCV. Und das nicht ohne Grund: OpenCV bietet ein riesiges Spektrum an Funktionen und Algorithmen, von einfachen Bildoperationen hin zu ausgereiften Algorithmen der Gesichtserkennung, Bildsegmentierung und Objekterkennung. Auch Maschinelles Lernen und Deep Learning sind im Funktionsumfang von OpenCV enthalten. 

Jedoch wurde OpenCV nun einmal in C++ geschrieben und kann somit nicht direkt auf Clientseite implementiert werden. Aber nicht immer ist die Option sinnvoll, Aufgaben der Bildverarbeitung auf der Seite des Servers lösen zu lassen. Nicht immer steht die benötigte Rechenleistung zur Verfügung, auch die Latenz kann letztendlich je nach Anwendungsfalls ein Problem darstellen. Somit stellt sich einem die folgende Frage: Ist es möglich OpenCV auch im Browser des Clients zu lösen? Die Antwort: Ja! Mit openCV.js.

\subsection{Motivation}
Aber warum der Blog? Einst musste ich (wie du vielleicht auch) ein Problem der Bildverarbeitung lösen, welches zwingend im Webkontext stattfinden sollte. Die Anforderungen waren klar: die Bildverarbeitung sollte direkt im Browser des Clients stattfinden, ohne dass der Nutzer eine zusätzliche Software installieren muss. 

Die Bibliothek openCV war natürlich die erste Wahl, jedoch stand ich nun genau vor diesem Problem: wie bekomme ich openCV, eine Bibliothek, die ursprünglich in C++ geschrieben ist, in meiner Webanwendung zum Laufen? Meine Lösung war natürlich openCV.js.

Als relativ neuer Bestandteil des openCV-Projektes, ist openCV.js eine JavaScript-Portierung der OpenCV-Bibliothek. Sie ermöglicht es, OpenCV-Funktionen direkt im Browser auszuführen, ohne dass der Nutzer eine zusätzliche Software installieren muss. Somit können wir die volle Bandbreite der OpenCV-Funktionen nutzen, ohne auf die Vorteile einer Webanwendung verzichten zu müssen.

Obwohl es sich zunächst als die beste Lösung angehört hat, war die Nutzung von openCV.js jedoch kein Selbstläufer. Wie du wahrscheinlich bereits mit Schrecken festgestellt hast, ist die offizielle Dokumentation nur für die C++ Schnittstelle geschrieben, und Tutorials für openCV.js sind leider rar gesät. Die JavaScript-SChnittstelle von openCV ist nun mal die neuste Änderung des mittlerweile 20 Jahre alten Projektes und somit noch nicht so ausgereift und erprobt wie die anderen Schnittstellen.

Genau aus diesen Gründen habe ich mich für das Schreiben dieses Devblogs entschieden. Ich möchte meine Erfahrungen und Erkenntnisse teilen, um anderen Entwicklern zu helfen, die vor dem gleichen Problem stehen. Ich möchte zeigen, dass es gar nicht so kompliziert ist, openCV.js in einer Webanwendung zu nutzen und wie mächtig die Bibliothek tatsächlich ist. Ja es gibt ein paar Eigenheiten und Macken, aber genau diese werde ich erläutern, damit du nicht die gleichen Fehler machst wie sie ich einst gemacht habe.

\subsection{Zielsetzung}
Kommen wir nun zu den eigentlichen Fragen: Was möchte ich erreichen? Was möchte ich entwickeln? Wie bereits oben erwähnt, möchte ich eine Webanwendung erstellen, welche durch Nutzung von openCV.js Probleme der Bildverarbeitung und Bilderkennung lösen kann. Das konkrete Problem sollte dabei sowohl häufig genutzte Methoden der Bildverarbeitung abdecken, als auch leicht verständlich und reproduzierbar sein. Ich möchte, dass du die Anwendung einfach nachbauen und selber ausprobieren kannst, um die Funktionsweise von openCV.js direkt durch eigene Erfahrungen zu erlernen.

Nach einigen Überlegungen bin ich auf die Idee gekommen, eine Webanwendung zu entwickeln, welche Münzen erkennen und deren Wert bestimmen kann. Warum ausgerechnet Münzen? Zum einen ist es ein angemessen komplexes Problem, welches gleich mehrere unterschiedliche Aspekte der Bildverarbeitung abdeckt. Zum anderen ist es eine Problemstellung, welche sich leicht auf einem anderen System reproduzieren lässt. Alles was du brauchst, sind lediglich eine Webkamera und ein paar Münzen.

Interessiert? Dann lass uns anfangen! Im meinem nächsten Beitrag werde ich direkt die grundlegende Webseitenstruktur und die Einbindung von openCV.js erläutern. 

Bis dahin!