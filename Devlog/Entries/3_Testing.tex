\section{Eingriffe zur Laufzeit}

\subsection{Die Bedeutung von einfachem Testing}
Meine nächste Aufgabe, die Implementierung der Kreiserkennung, erfordert eine Menge an Feintuning und Anpassungen. Jeder Parameter muss sorgfältig gewählt werden, um ein optimales Ergebnis zu erzielen. Zudem kann ich vorhinein nicht ermittelt werden, welche Parameter die besten sind. Es ist also ein iterativer Prozess, bei dem ich die Parameter anpasse, das Ergebnis betrachte und dann erneut die Parameter anpasse.

Nun stelle man sich vor, man müsste für jede kleine Änderung die Webseite neu laden, die Kamera neu ausrichten und die Münzen neu platzieren. Um diesen Prozess deutlich zu beschleunigen, habe ich mich dazu entschieden, einige Eingriffe zur Laufzeit zu implementieren. Diese Eingriffe sollen es mir ermöglichen, die Parameter der Kreiserkennung und weiterer openCV Funktionen direkt zur Laufzeit zu verändern, ohne die Webseite neu laden zu müssen, und somit schnell die besten Parameter zu finden.

\subsection{Variablen-Slider}
Meine Idee war es, für jede Variable im JavaScript-Code einen Slider zu erstellen zu können, mit dem ich schnell und einfach den Wert der Variable verändern kann. Dieser Slider soll dann den Wert der Variable anzeigen und bei Veränderung den Wert der Variable ändern.

Für die Slider habe ich die Bibliothek \href{https://refreshless.com/nouislider/}{noUISlider} verwendet.  Über die data-Attribute von den HTML-Elementen kann unkompliziert der Wertebereich und die Schrittweite des Sliders festgelegt werden.

Nun muss aber standardmäßg ein Event-Listener für jeden Slider erstellt werden, der bei Veränderung den Wert einer Variable ändert. Da ich jedoch nicht für jede Variable einen eigenen Event-Listener erstellen möchte, habe ich mich für einen anderen Ansatz entschieden. Stattdessen soll man direkt im HTML-Element des Sliders angeben können, welche Variable durch diesen Slider verändert werden soll.

Hierfür benötigen wir zum Ersten ein eigenes data-Attribute von HTML, in welchem der Name der zu verändernen Varaible angegeben wird. Zum Zweiten benötigen wir eine Funktion, die alle Slider-Elemente durchgeht und für jedes Element den Event-Listener erstellt. Anschließend braucht es noch eine Möglichkeit den String-Namen der Variable in eine Referenz auf die Variable umzuwandeln.

Die Event-Listener Funktion sieht dann wie folgt aus:

\begin{lstlisting}[style=JavaScript]
function InitSliders(){
    // Alle Slider-Container selektieren
    const sliderContainers = document.querySelectorAll('.sliderContainer');

    sliderContainers.forEach(container => {
        const sliderElement = container.querySelector('.slider');
        const valueElement = container.querySelector('.sliderValue');

        const min = parseFloat(container.dataset.min);
        const max = parseFloat(container.dataset.max);
        const step = parseFloat(container.dataset.step);
        const var1 = container.dataset.var1;
        const var2 = container.dataset.var2;
        const isRange = container.dataset.range === "true"; // Überprüft, ob Bereichsmodus aktiv ist
        //console.log("Data for slider: min: " + min + " max: " + max + " step: " + step + " var1: " + var1 + " var2: " + var2 + " isRange: " + isRange);

        // Startwerte auslesen
        let startValues = [];
        if (isRange) {
            // Bereichsmodus: Startwerte aus den <span>-Elementen lesen
            const spanValues = valueElement.querySelectorAll('span');
            startValues = Array.from(spanValues).map(span => parseFloat(span.textContent));
            if (startValues.length !== 2) {
                // Fallback: Standardwerte in der Mitte des Bereichs
                startValues = [min + (max - min) / 3, max - (max - min) / 3];
            }
        } else {
            // Einzelregler: Einzelwert auslesen
            startValues = [parseFloat(valueElement.textContent) || (min + max) / 2];
        }

        // Slider erstellen
        noUiSlider.create(sliderElement, {
            start: startValues,
            range: {
                'min': min,
                'max': max
            },
            step: step,
            connect: isRange ? true : [true, false] // Verbindet die Regler bei Range-Modus
        });

        SetSliderValueFromCookie(container.id);

        // Update-Event
        sliderElement.noUiSlider.on('update', (values, handle) => {
            if (isRange) {
                // Bereichsmodus: Werte in <span>-Elementen aktualisieren
                if(step >= 1){
                    valueElement.textContent = values.map(value => Math.round(value)).join(' - ');
                }else{
                    valueElement.textContent = values.map(value => value).join(' - ');
                }

                //TODO: Beide Variablen aktualisieren
                if(var1 !== undefined && var2 !== undefined){
                    UpdateVariable(var1, values, 0);
                    UpdateVariable(var2, values, 1);
                }else{
                    console.log("var1 or var2 is undefined. Slider will not change any variables");
                }


            } else {
                // Einzelregler: Textinhalt aktualisieren
                if(step >= 1){
                    valueElement.textContent = Math.round(values[handle]);
                }else{
                    valueElement.textContent = values[handle];
                }

                if(var1 !== undefined){
                    // Variable aktualisieren
                    UpdateVariable(var1, values, handle);
                }else{
                    console.log("var1 is undefined. Slider will not change any variables");
                }

            }

            // Cookie setzen
            SetSliderCookie(container.id, values);
        });
    });

    console.log("Sliders initialized");
}
\end{lstlisting}

Die Funktion UpdateVariable() wandelt schließlich den String-Namen der Variable in eine Referenz um und ändert den Wert der Variable. Hierfür wird die eval()-Funktion verwendet, die den String als JavaScript-Code interpretiert.

\begin{lstlisting}[style=JavaScript]
function UpdateVariable(varName, values, handle){
    //check if variable exists
    if (eval('typeof ' + varName) === 'undefined') {
        console.warn('Variable '+varName+' does not exist');
        return;
    }

    //update variable
    if(eval('typeof ' + varName) === 'number'){
        eval(varName + ' = ' + parseFloat(values[handle]));
    }else if(eval('typeof ' + varName) === 'string'){
        eval(varName + ' = ' + values[handle]);
    }
}
\end{lstlisting}

Wie du vielleicht gesehen hast, gibt es auch noch die Funktionen SetSliderValueFromCookie() und SetSliderCookie(). Diese Funktionen dienen dazu, die Werte der Slider in Cookies zu speichern und beim Laden der Webseite wiederherzustellen. So behalten die Slider ihre eingestellten Werte auch nach einem Neuladen der Webseite.

\subsection{Cookies}

Cookies sind kleine Textdateien, die auf dem Computer des Benutzers gespeichert werden. Sie dienen dazu, Informationen über den Benutzer zu speichern, um die Benutzererfahrung zu verbessern. In meinem Fall möchte ich die Werte der Slider speichern, um sie beim Neuladen der Webseite wiederherzustellen.

Jedoch ist das Speichern von Werten in Cookies nicht ganz trivial: Pro Cookie kann nur ein einziger String mit einer maximalen Länge von 4096 Bytes gespeichert werden. Da die Werte der Slider jedoch Arrays von Zahlen sind, müssen diese erst in einen String umgewandelt werden. Hierfür habe ich mich für die JSON.stringify()-Funktion entschieden, die ein JavaScript-Objekt in einen JSON-String umwandelt. Die Funktion zum schreiben des Cookies sieht dann wie folgt aus:

\begin{lstlisting}[style=JavaScript]
function SetSliderCookie(sliderId, value) {
    // Holen des bestehenden Cookie-Werts
    let sliderValues = GetSliderCookie();

    // Setze den Wert des Sliders im Cookie
    sliderValues[sliderId] = value;

    // Setze den Cookie mit den neuen Werten
    const expires = new Date();
    expires.setDate(expires.getDate() + 7); // Cookie läuft in 7 Tagen ab
    document.cookie = `sliderValues=${JSON.stringify(sliderValues)}; expires=${expires.toUTCString()}; path=/;`;
}
\end{lstlisting}

Zunächst erstelle ich ein Objekt "sliderValues" aus dem Cookie, falls es bereits existiert. Anschließend füge ich den neuen Wert des Sliders hinzu und speichere das Objekt als JSON-String im Cookie. Hierfür muss zudem ein Ablaufdatum für den Cookie festgelegt werden, in meinem Fall läuft der Cookie nach 7 Tagen ab. Der Cookie sieht dann wie folgt aus:

\begin{lstlisting}[style=JavaScript]
{"slider1":["5100.00"],"slider2":["3900.00"],"slider3":["7.00"],"slider4":["3.00"],"sliderValue3":["230.00"],"sliderValue4":["129.00"]}
\end{lstlisting}

Um die Daten aus dem Cookie wiederherzustellen, muss in der Funktion GetSliderCookie() der Cookie ausgelesen und in ein JavaScript-Objekt umgewandelt werden. Eine saubere Lösung, welche selbst beim Vorhandensein von weiteren Cookies funktioniert, könnte wie folgt aussehen:

\begin{lstlisting}[style=JavaScript]
function GetSliderCookie(){
    const value = `; ${document.cookie}`;
    const parts = value.split(`; sliderValues=`);
    if (parts.length === 2) {
        const cookieValue = parts.pop().split(';').shift();
        return JSON.parse(cookieValue);  // Parsen des JSON-Strings
    }
    return {};  // Falls der Cookie nicht existiert, ein leeres Objekt zurückgeben
}
\end{lstlisting}

Lass mich diese Funktion kurz erklären: Zunächst wird der Cookie-Wert aus dem document.cookie-String extrahiert. Die \enquote{cookie} Eigenschaft des Objekts \enquote{document} enthält alle Cookies, die auf der aktuellen Seite gesetzt wurden. Dem String wird ein Semikolon und ein Leerzeichen vorangestellt, um sicherzustellen, dass der Cookie-Wert korrekt extrahiert wird, selbst wenn noch andere Cookies vorhanden sind. Nun wird über value.split('; sliderValues=') der String zweigeteilt, der erste Teil, welcher alles vor dem Cookie-Wert enthält, wird per pop() verworfen. Der zweite Teil, welcher den Cookie-Wert inklusive möglicher weiterer Cookies enthält, muss noch weiter verarbeitet werden. Hierfür wird der String per split(';') wieder in ein Array geteilt und mit shift() das erste Element, welches den Cookie-Wert enthält, extrahiert. Nun erst können wir sicher sein, dass mögliche Cookies vor und nach unserem Cookie-Wert entfernt wurden. Der Cookie-Wert wird schließlich per JSON.parse() in ein JavaScript-Objekt umgewandelt und zurückgegeben.

Zu guter Letzt braucht es noch eine Funktion, welche beim Start der Webseite die Werte der Slider aus dem Cookie wiederherstellt. Diese Funktion sieht wie folgt aus:

\begin{lstlisting}[style=JavaScript]
function SetSliderValueFromCookie(containerID){
    const sliderValues = GetSliderCookie();
    const sliderContainer = document.getElementById(containerID);

    if (!sliderContainer) {
        console.error('Slider container not found');
        return;
    }

    const slider = sliderContainer.querySelector('.slider');
    const sliderValueElement = sliderContainer.querySelector('.sliderValue');

    if (!slider || !sliderValueElement) {
        console.error('Slider or value element not found');
        return;
    }

    const savedValue = sliderValues[containerID];

    if (savedValue !== undefined) {
        // Den Slider auf den gespeicherten Wert setzen
        slider.noUiSlider.set(savedValue);

        // Anzeige des aktuellen Werts aktualisieren
        sliderValueElement.textContent = Array.isArray(savedValue) ? savedValue.join(' - ') : savedValue;
    }else{
        console.log("No saved value for " + containerID + " found in cookie");
    }
}
\end{lstlisting}

Nun habe ich endlich eine solide Möglichkeit geschaffen, schnell und unkompliziert Variablen zur Laufzeit zu verändern und diese Änderungen sogar über einen Neustart der Webseite hinweg zu speichern. Dies wird mir hoffentlich eine Menge Zeit und Nerven sparen, wenn ich nun als nächstes mit der Implementierung der Kreiserkennung beginne.